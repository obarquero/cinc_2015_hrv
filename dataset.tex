\section{Dataset}
 
This study recruited eight Spanish male triathletes from Seville Metropolitan Area. For participating in this investigation, all subjects signed an informed written consent based on the Declaration of Helsinki. They also followed the recommendations for HRV data acquisition, by avoiding the use of stimulant substances the day of the trial, having the last meal at least two hours before performing the test, and not exercising the day before testing. Inclusion criteria were the following: (a) to be male and amateur triathlete; (b) not having any disease or medication that could affect the outcomes; (c) to train more than ten hours per week; and (d) to be competing actively in amateur circuits. All measurements of the study were made while subjects were sitting on a cycle ergometer.

The protocol included 5 minutes of HR recording in resting conditions, immediately before exercising, in order to provide the HRV basal data. Afterwards, subjects started an incremental exercise test which comprised 4 consecutive phases of 4 minutes. In the first three phases, subjects cycled at 50 rpm with 1, 2, and 3 Kp of load. The last part of the test was the so-called `All Out', in which subjects cycled as fast as they could with 5 Kp of load during 4 minutes for achieving maximal exercise capacity. HR was recorded in the 'All Out' step. The exercise test was followed by 5 minutes of recovery, in which subjects remained seated on the cycle ergometer, but without cycling. HR was also recorded in the recovery stage. 

RR intervals were collected using a Firstbeat Bodyguard (Firstbeat Technologies Oy$^{TM}$, Jyv\"askyl\"a�, Finland) heart monitor, with sampling frequency of 1000 Hz. The recordings were preprocessed to exclude artifacts by eliminating RR intervals lower than $200$ ms and greater than $2000$ ms, as well as those which differed more than $20\%$ from the previous and the subsequent RR intervals~\cite{Malik89}. Removed RR intervals were replaced by conventional spline interpolation.  
