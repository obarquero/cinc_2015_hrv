
\section{Introduction}

A characteristic of physiologic systems is their deep complexity,  arising from internal interactions and regulatory feedback loops which operate over a wide range of temporal and spatial scales~\cite{Goldberger02}. Heart Rate Variability (HRV) is a relevant marker of the Autonomic Nervous System (ANS) control on the heart, and it has been proposed for risk stratification of lethal arrhythmias after acute myocardial infarction, as well as for prognosis of sudden cardiac death events~\cite{Malik89,Huikuri00}. A wide number of HRV indices have been proposed in the literature, many studies suggest that nonlinear methods are better suited to extract relevant information  from HRV signal in terms of complexity. Nonlinear indices rely on the idea that fluctuations in the RR intervals may reveal characteristics from complex dynamic systems, and, under this assumption, healthy states will correspond to more complex patterns than pathological states~\cite{Huikuri00,goldberger91,pincus91}.  Furthermore, many experts claim that no single index should be used to assess the complexity of physiologic systems, instead of that, a set of metrics is needed to measure different aspects of the complicated behavior of physiologic systems~\cite{Goldberger02}.

It has been shown that heart rate increases during dynamic exercise due to both a parasympathetic withdrawal and augmented sympathetic activity. The relative role of these two drives depends on the exercise intensity~\cite{Aubert03}. Parasympathetic activation is considered to be the main mechanism underlying the Heart Rate Recovery (HRR) after exercise. HRR immediately after high intensity exercise is used as a marker of physical condition, and also some studies have associated decreased HRR to cardiovascular risk increase~\cite{Cabrera97}. In order to improve the understanding on the ANS changes in this context, HRV has been widely assessed by means of  time domain and frequency domain indices, however nonlinear indices have recibed less attention.  

In this work we propose to extend the knowledge of HRR after maximal exercise using a set of nonlinear indices, which allow to measure different aspects of the underlying physiological mechanisms. The study focuses on HRV evolution before, during, and after an All Out Exercise Testing (AOET) in triathletes. Additionally, in order to quantify HRV Recovery (HRVR) after exercising, we propose a HRVR measure which can be used for any HRV index. 

The structure of the paper is as follows. First, the database and the HRV nonlinear indices are presented, and the rationale for the study is highlighted. Next, the data analysis is described in detail, and the results are presented. Finally, conclusions are summarized.
