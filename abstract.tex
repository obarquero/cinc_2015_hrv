
\begin{abstract}
\textbf{Introduction}. At high altitude there is a reduced oxygen pressure in the atmosphere which results in physiological changes. Heart Rate Variability (HRV) is a technique to quantify the autonomic nervous system regulation of the heart rate, allowing a noninvasive assessment in extreme environments.

\textbf{Aim}. The aim of this work was to assess the evolution of the HRV complexity during Kangchenjunga (8.586 m) climbing.

\textbf{Data}. Three climbers recorded their RR-interval time series every day during the expedition. We divided the data in different stages: Spain Baseline, Kathmandu Baseline, Acclimation Trekking, Kathmandu After Acclimation, Base Camp, Camp 1, Camp 2, Summit, Camp 3, and Kathmandu after expedition.


\textbf{Methods}. We assessed the complexity of HRV using sample entropy (SampEn) and normalized compression distance (NCD), a measure coming from Information Theory, which compares two arbitrary sequences and outputs the dissimilarity between them. This measure exploits linear and nonlinear relations in the data and allows the comparison of sequences of different sizes. We estimated the dissimilarity of every stage in the climb against the first day using NCD.

\textbf{Results}. From the beginning and during acclimation dissimilarity (NCD) increased and then decreased once the climbers were acclimated. Dissimilarity jumped up in Base Camp stage. Then dissimilarity decreased from this point until the end of the expedition. SampEn showed an irregular behaviour without a clear pattern.

\textbf{Conclusion}. NCD provided a method to assess the dissimilarity of HRV between different stages in  Kangchenjunga expedition climbing, and allowed to quantify the changes in HRV complexity.
\end{abstract}